\chapter{Automatentheorie}
\begin{flushleft}
    Die Automatentheorie befasst sich mit formalen Sprachen und Grammatiken.
    Es geht hauptsächlich darum diese Sprachen durch Automaten zu verarbeiten.
\end{flushleft}

\section{Endliche Automaten}
\subsection{Definition}
\begin{flushleft}
    Ein endlicher Automat wird durch so einen 5er-Tupel definiert:
    \begin{align}
        A=(\Sigma,Q,\delta,q_0,F)
    \end{align}
    \begin{enumerate}
        \item {
            Das Alphabet $\Sigma$ ist die Menge der erlaubten Eingabesymbole des Automaten.
            Bei endlichen Automaten muss diese Menge $\Sigma$ endlich sein.
        }
        \item {
            Die zweite Menge $Q$ des Automaten ist die Menge aller Zustände dieses Automaten.
            Die Anzahl der Zustände des Automaten muss endlich sein.
            Außerdem gilt $q_0 \in Q$, da $q_0$ der Startzustand des Automaten ist.
        }
        \item {
            Da der Automat endlich ist gibt es auch eine Menge $F$ von Endzuständen.
            $Q$ enthält alle Zustände, deshalb gilt $F \subseteq Q$.
        }
        \item {
            Übrig bleibt die Zustandsübergangsfunktion $\delta: Q \times \Sigma \rightarrow Q$.
            Diese Übergangsfunktion bestimmt welcher Zustand bei welcher bestimmten Eingabe folgt.
        }
    \end{enumerate}
    Endliche Automaten werden in Bezug auf Determinismus unterschieden.
    Ein Automat ist nicht deterministisch, wenn dieser von einem Zustand ausgehend, mehrere Übergänge
    für gleiche Zeichen definiert.
\end{flushleft}

\subsection{Transduktor oder Akzeptor}
\begin{flushleft}
    Während Transduktoren eine beliebige Ausgabe auf bestimmte Eingaben geben können, haben 
    Akzeptoren nur die Möglichkeit eine Eingabe zu akzeptieren oder abzulehnen.
    Transduktoren sind daher etwas komplizierter als Akzeptoren, sie werden in der Regel durch einen 7er-Tupel definiert:
    \begin{align}
        A=(\Sigma,\Gamma,Q,\delta,q_0,F,\omega)
    \end{align}
    Neu sind hier bloß $\Gamma$, die Menge der Ausgabesymbole und $\omega: Q \times \Sigma \rightarrow \Gamma^*$, die Ausgabefunktion.
\end{flushleft}

\subsection{Beispiel}
\begin{flushleft}
    Als Beispiel für einen deterministischen endlichen Automaten ist hier der Übergangsgraph eines
    Automaten, der nur die Zahl 42 akzeptiert:
    \begin{center}
    \begin{tikzpicture}[->,
        shorten >=5pt,
        node distance=2.5cm,
        semithick]
        \node[initial,state] (R) {$q_0$};
        \node[state] (S) [right of=R] {$q_1$};
        \node[state,accepting] (T) [right of=S] {$q_2$};
        \path (R) edge node [below] {4} (S)
              (S) edge node [below] {2} (T);
    \end{tikzpicture}
    \end{center}
    Formal ist dieser Automat wie folgt definiert:
    \begin{align}
        A &= (\Sigma,Q,\delta,q_0,F) \\
        \Sigma &= \{4,2\} \\
        Q &= \{q_0,q_1,q_2\} \\
        \delta &= \{\langle \langle q_0, 4 \rangle, q_1 \rangle, \langle \langle q_1, 2 \rangle, q_2 \rangle \} \\
        F &= \{q_2\}
    \end{align}
    Die Zustände dieses Automaten lassen sich auch etwas anschaulicher in einer Übergangstabelle darstellen:
    \begin{center}
        \begin{tabular}{|c|c|c|}
            \hline
            $\delta$ & 4 & 2 \\
            \hline
            $q_0$ & $q_1$ & $q_0$ \\
            \hline
            $q_1$ & $q_1$ & $q_2$ \\
            \hline
            $q_2$ & $q_2$ & $q_2$ \\
            \hline
        \end{tabular}
    \end{center}
\end{flushleft}

\subsection{Potenzmengenkonstruktion}
\begin{flushleft}
    Die Funktionsweise von deterministischen endlichen Automaten lässt sich relativ einfach erkennen.
    Bei nicht-deterministischen endlichen Automaten ist das aber nicht so, deshalb kann man eine Potenzmengenkonstruktion 
    durchführen um einen beliebigen nicht-deterministischen endlichen Automaten (NEA) in einen deterministischen endlichen 
    Automaten (DEA) umzuwandeln.
    \subsubsection{Beispiel}
    Dieser NEA soll jetzt beispielhaft in einen DEA umgewandelt werden:
    \begin{center}
    \begin{tikzpicture}[->,
        shorten >=5pt,
        node distance=2.5cm,
        semithick]
        \node[initial,state] (R) {$q_0$};
        \node[state,accepting] (S) [right of=R] {$q_1$};
        \path (R) edge [loop below,below] node {a} (R)
              (R) edge [below] node {a} (S)
              (S) edge [loop below,below] node {b} (S);
    \end{tikzpicture}
    \end{center}
    Der erste Schritt ist hierbei das Bestimmen der Zustandsmenge $Q$. In diesem Beispiel gilt:
    \begin{align}
        Q=\{q_0,q_1\}
    \end{align}
    Um nun einen DEA aus diesem NEA zu machen muss die Potenzmenge dieser Zustandsmenge bestimmt werden.
    Dieser Prozess könnte bereits aus der Mengenlehre bekannt sein, $Q'$ ist hier die Potenzmenge von $Q$:
    \begin{align}
        Q &=\{q_0,q_1\} \\
        Q' &=\{\emptyset,\{q_0\},\{q_1\},\{q_0,q_1\}\}
    \end{align}
    Das Symbol $\emptyset$ steht für die leere Menge $\{\}$. Diese Potenzmenge $Q'$ ist die Zustandsmenge 
    des neuen DEA. Um später einen schöneren Übergangsgraph zeichnen zu können wird diese Menge $Q'$ vereinfacht 
    (durch die zweite Änderung sollte $Q'$ eigentlich zu $Q''$ werden, hier bleibt der Name jedoch gleich):
    \begin{align}
        Q' =\{\emptyset,q_0,q_1,q_{01}\}
    \end{align}
    Wichtig zu beachten ist, dass $q_1 \neq q_{01}$ ist, da $q_1=\{q_1\}$ und $q_{01}=\{q_0,q_1\}$ ist.
    Mit der Menge $Q'$ als Zustandsmenge unseres DEA, sieht der erste Entwurf des Übergangsgraphen etwa so aus:
    \begin{center}
    \begin{tikzpicture}[->,
        shorten >=5pt,
        node distance=2.5cm,
        semithick]
        \node[state] (0) {$q_0$};
        \node[state] (1) [above right=0.65cm and 1.5cm of 0] {$q_1$};
        \node[state] (2) [below right=0.65cm and 1.5cm of 0] {$q_{01}$};
        \node[state] (E) [right=3.75cm of 0] {$\emptyset$};
    \end{tikzpicture}
    \end{center}
    Wir beginnen damit die Start- und Endzustände zu bestimmen. Den Startzustand (hier $q_0$) können 
    wir einfach übernehmen. Bei den Endzuständen ist das Vorgehen etwas anders, jeder Zustand, dessen Menge 
    einen Endzustand des NEA enthält ist ein Endzustand des DEA. In diesem Beispiel sind also $q_1$ und $q_{01}$
    Endzustände:
    \begin{center}
    \begin{tikzpicture}[->,
        shorten >=5pt,
        node distance=2.5cm,
        semithick]
        \node[initial,state] (0) {$q_0$};
        \node[accepting,state] (1) [above right=0.65cm and 1.5cm of 0] {$q_1$};
        \node[accepting,state] (2) [below right=0.65cm and 1.5cm of 0] {$q_{01}$};
        \node[state] (E) [right=3.75cm of 0] {$\emptyset$};
    \end{tikzpicture}
    \end{center}
    Um die Übergänge des DEA zu bestimmen, zeichnen wir zu erst die Übergangstabelle des NEA:
    \begin{center}
        \begin{tabular}{|c|c|c|}
            \hline
            $\delta$ & a & b \\
            \hline
            $q_0$ & $\{q_0,q_1\}$ & $\emptyset$ \\
            \hline
            $q_1$ & $\emptyset$ & $q_1$ \\
            \hline
            $\emptyset$ & $\emptyset$ & $\emptyset$ \\
            \hline
        \end{tabular}
    \end{center}
    Hier ist relativ leicht zu erkennen, dass $\{q_0,q_1\}$ der Grund für den fehlenden 
    Determinismus des NEA ist. Genau dieser Ausdruck sollte uns jedoch aus der vereinfachten Potenzmenge 
    $Q'$ bekannt vorkommen, $\{q_0,q_1\}=q_{01}$. Um einen DEA aus dem NEA zu machen muss man hier also bloß 
    $\{q_0,q_1\}$ durch den neuen Zustand $q_{01}$ austauschen:
    \begin{center}
        \begin{tabular}{|c|c|c|}
            \hline
            $\delta$ & a & b \\
            \hline
            $q_0$ & $q_{01}$ & $\emptyset$ \\
            \hline
            $q_1$ & $\emptyset$ & $q_1$ \\
            \hline
            $q_{01}$ & ? & ? \\
            \hline
            $\emptyset$ & $\emptyset$ & $\emptyset$ \\
            \hline
        \end{tabular}
    \end{center}
    Um einen kompletten DEA zu konstruieren muss jedoch noch definiert werden, wie sich der DEA bei $q_{01}$ verhält.
    Dieses Verhalten muss aus dem NEA genommen werden. Da $q_{01}$ die Menge von $q_0$ und $q_1$ ist, kombiniert $q_{01}$
    das Verhalten dieser beiden Zustände. Bei einem $a$ bleibt der DEA also im Zustand $q_{01}$ und bei einem $b$ wandert 
    der DEA in den Zustand $q_1$:
    \begin{center}
        \begin{tabular}{|c|c|c|}
            \hline
            $\delta$ & a & b \\
            \hline
            $q_0$ & $q_{01}$ & $\emptyset$ \\
            \hline
            $q_1$ & $\emptyset$ & $q_1$ \\
            \hline
            $q_{01}$ & $q_{01}$ & $q_1$ \\
            \hline
            $\emptyset$ & $\emptyset$ & $\emptyset$ \\
            \hline
        \end{tabular}
    \end{center}
    Basierend auf dieser Übergangstabelle kann man diesen Übergangsgraphen zeichnen:
    \begin{center}
    \begin{tikzpicture}[->,
        shorten >=5pt,
        node distance=2.5cm,
        semithick]
        \node[initial,state] (0) {$q_0$};
        \node[accepting,state] (1) [above right=0.65cm and 1.5cm of 0] {$q_1$};
        \node[accepting,state] (2) [below right=0.65cm and 1.5cm of 0] {$q_{01}$};
        \node[state] (E) [right=3.75cm of 0] {$\emptyset$};
        \path (0) edge [below] node {a} (2)
              (0) edge [above] node {b} (E)
              (1) edge [above] node {a} (E)
              (1) edge [loop left] node {b} (1)
              (2) edge [loop below] node {a} (2)
              (2) edge [below left] node {b} (1)
              (E) edge [loop below] node {a,b} (E);
    \end{tikzpicture}
    \end{center}
    \pagebreak
    Um diesen unordentlichen Graphen zu bereinigen kann man hier den sogenannten Papierkorbzustand $\emptyset$
    entfernen. Die Übergangstabelle würde dann so aussehen:
    \begin{center}
        \begin{tabular}{|c|c|c|}
            \hline
            $\delta$ & a & b \\
            \hline
            $q_0$ & $q_{01}$ & $q_0$ \\
            \hline
            $q_1$ & $q_1$ & $q_1$ \\
            \hline
            $q_{01}$ & $q_{01}$ & $q_1$ \\
            \hline
        \end{tabular}
    \end{center}
    Hier der übersichtliche Übergangsgraph:
    \begin{center}
    \begin{tikzpicture}[->,
        shorten >=5pt,
        node distance=2.5cm,
        semithick]
        \node[initial,state] (0) {$q_0$};
        \node[accepting,state] (2) [right of=0] {$q_{01}$};
        \node[accepting,state] (1) [right of=2] {$q_1$};
        \path (0) edge [below] node {a} (2)
            (0) edge [loop below] node {b} (0)
            (1) edge [loop below] node {a,b} (1)
            (2) edge [loop below] node {a} (2)
            (2) edge [below] node {b} (1);
    \end{tikzpicture}
    \end{center}
\end{flushleft}
