\chapter{Automatentheorie}
\begin{flushleft}
    Die Automatentheorie befasst sich mit formalen Sprachen und Grammatiken.
    Es geht hauptsächlich darum diese Sprachen durch Automaten zu verarbeiten.
\end{flushleft}

\section{Automatentypen}
\subsection{Endliche Automaten}
\subsubsection{Definition}
\begin{flushleft}
    Ein endlicher Automat wird durch so einen 5er-Tupel definiert:
    \begin{align}
        A=(\Sigma,Q,\delta,q_0,F)
    \end{align}
    \begin{enumerate}
        \item {
            Das Alphabet $\Sigma$ ist die Menge der erlaubten Eingabesymbole des Automaten.
            Bei endlichen Automaten muss diese Menge $\Sigma$ endlich sein.
        }
        \item {
            Die zweite Menge $Q$ des Automaten ist die Menge aller Zustände dieses Automaten.
            Die Anzahl der Zustände des Automaten muss endlich sein.
            Außerdem gilt $q_0 \in Q$, da $q_0$ der Startzustand des Automaten ist.
        }
        \item {
            Da der Automat endlich ist gibt es auch eine Menge $F$ von Endzuständen.
            $Q$ enthält alle Zustände, deshalb gilt $F \subseteq Q$.
        }
        \item {
            Übrig bleibt die Zustandsübergangsfunktion $\delta: Q \times \Sigma \rightarrow Q$.
            Diese Übergangsfunktion bestimmt welcher Zustand bei welcher bestimmten Eingabe folgt.
        }
    \end{enumerate}
    Endliche Automaten werden in Bezug auf Determinismus unterschieden.
    Ein Automat ist nicht deterministisch, wenn dieser von einem Zustand ausgehend, mehrere Übergänge
    für gleiche Zeichen definiert.
\end{flushleft}

\subsubsection{Transduktor oder Akzeptor}
\begin{flushleft}
    Während Transduktoren eine beliebige Ausgabe auf bestimmte Eingaben geben können, haben 
    Akzeptoren nur die Möglichkeit eine Eingabe zu akzeptieren oder abzulehnen.
    Transduktoren sind daher etwas komplizierter als Akzeptoren, sie werden in der Regel durch einen 7er-Tupel definiert:
    \begin{align}
        A=(\Sigma,\Gamma,Q,\delta,q_0,F,\omega)
    \end{align}
    Neu sind hier bloß $\Gamma$, die Menge der Ausgabesymbole und $\omega: Q \times \Sigma \rightarrow \Gamma^*$, die Ausgabefunktion.
\end{flushleft}

\subsubsection{Beispiel}
\begin{flushleft}
    Als Beispiel für einen deterministischen endlichen Automaten ist hier der Übergangsgraph eines
    Automaten, der nur die Zahl 42 akzeptiert:
    \begin{center}
    \begin{tikzpicture}[->,
        shorten >=5pt,
        node distance=2.5cm,
        semithick]
        \node[initial,state] (R) {$q_0$};
        \node[state] (S) [right of=R] {$q_1$};
        \node[state,accepting] (T) [right of=S] {$q_2$};
        \path (R) edge node [below] {4} (S)
              (S) edge node [below] {2} (T);
    \end{tikzpicture}
    \end{center}
    Formal ist dieser Automat wie folgt definiert:
    \begin{align}
        A &= (\Sigma,Q,\delta,q_0,F) \\
        \Sigma &= \{4,2\} \\
        Q &= \{q_0,q_1,q_2\} \\
        \delta &= \{\langle \langle q_0, 4 \rangle, q_1 \rangle, \langle \langle q_1, 2 \rangle, q_2 \rangle \} \\
        F &= \{q_2\}
    \end{align}
\end{flushleft}
