\chapter{Datenstrukturen}
\begin{flushleft}   
    Datenstrukturen speichern Daten, damit später ohne Probleme auf diese Daten zugegriffen werden kann.
    Man nennt sie Daten\textbf{\textit{strukturen}}, da die Daten, die gespeichert werden auf eine bestimmte Art und Weise angeordnet (strukturiert) werden.
\end{flushleft}

\section{Statisch}
\begin{flushleft}
    Statische Datenstrukturen sind -- wie der Name schon sagt -- bestimmt.
    Sie sind relativ fest angeordnet und können schlecht bis gar nicht verändert werden.
\end{flushleft}

\subsection{Array}
\begin{flushleft}
    Arrays sind im Prinzip Listen, die eine feste Größe haben.
    Sie speichern Elemente eines Typs.
    Das bedeutet, dass ein einzelnes Array nicht dafür gedacht ist Integer und Strings zu speichern. \\
    Um Arrays vernünftig zu verstehen sieht man hier ein Beispiel für ein Array, das Integer speichert:
\end{flushleft}

\begin{center}
\begin{tikzpicture}
    \matrix (A) [matrix of nodes, nodes={draw, minimum size=8mm},column sep=-\pgflinewidth]{
        3 & 4 & 2 & 1 & 4 & 1 & 2 & 3 \\
    };
\end{tikzpicture}
\end{center}

\begin{flushleft}
    Arrays weisen jedem Element, das sie speichern einen \textit{Index} zu.
    Indices (Mehrzahl eines Index) starten in der Regel bei $0$.
    Das erste Element des Beispiel-Arrays hat den Wert $3$ und liegt beim Index $0$.
    Hier ein Beispiel zur Erstellung und Nutzung eines Arrays in Java.
    \begin{lstlisting}
        // Array.java
        
        // Unser Beispiel-Array erstellen
        int[] array = {3,4,2,1,4,1,2,3};

        // Die Laenge unseres Beispiel-Arrays abrufen
        int arrLen = array.length;
        
        // Den Wert des ersten Elements (3) in die Variable firstElem speichern
        int firstElem = array[0];

        // Ein Array erstellen, das vier mal den Wert 0 speichert
        int[4] array1;
    \end{lstlisting}
\end{flushleft}

\section{Dynamisch}
\begin{flushleft}   
    Dynamische Datenstrukturen sind -- vorallem im Vergleich zu statischen Datenstrukturen -- agil.
    Es ist sehr viel leichter Elemente zu entfernen oder hinzuzufügen, da die Größe von dynamischen Datenstrukturen nicht fest ist.
\end{flushleft}

\subsection{Queue}
\begin{flushleft}   
    Queue bedeutet übersetzt Warteschlange oder Schlange.
    Genau nach diesem Prinzip funktioniert auch eine Queue. \\
    Ein Mensch stellt sich in eine Schlange hinter andere Menschen.
    Vor ihm werden viele Leute aufgerufen, bis er dran kommt.
    Jede Person in der Warteschlange wird nacheinander abgearbeitet.
    Die erste Person, die sich anstellt, wird auch als erstes aufgerufen. \\
    Das nennt man auch \textit{First-In-First-Out} (\textit{FIFO}) Prinzip.
    Hier ein Bild zu der Erklärung:
\end{flushleft}

\begin{center}
\begin{tikzpicture}[draw,minimum width=1cm,minimum height=0.5cm]
    \node[draw] (in) at (-1,2) {\text{Neue Person}};
    \node[draw] (out) at (1,-2) {\text{Aufgerufener}};
    \matrix (queue)[matrix of nodes,nodes={draw, nodes={draw}}]{
         4. \\ 3. \\ 2. \\ 1. \\
     };

    \draw[-latex] (0.25,-1) .. controls (0.25,-1.25) and (1,-1.25) .. (out.north);
    \draw[-latex] (in.south) .. controls (-1,1.5) and (-0.25,1.5) .. (-0.25,1);
\end{tikzpicture}
\end{center}

\begin{flushleft}
    Das Aufrufen einer Person, entfernt diese aus der Queue.
    Dieser Prozess wird auch \textit{dequeue} genannt.
    Wenn sich eine neue Person anstellt, wird sie der Queue hinzugefügt.
    Das nennt man \textit{enqueue}. \\
    Weitere typische Operationen einer Queue sind:
    \begin{enumerate}
        \item {
                Das erste Element einer Queue aufrufen mit \textit{front}
            }
        \item {
                Prüfen ob die Queue leer ist mit \textit{isEmpty}
            }
    \end{enumerate}
    Eine allgemeine Queue würde also so aussehen:
\end{flushleft}

\begin{center}
\begin{tikzpicture}[draw,minimum width=1cm,minimum height=0.5cm]
    \node[draw] (in) at (-1,2) {\textit{enqueue}};
    \node[draw] (out) at (1,-2) {\textit{dequeue}};
    \matrix (queue)[matrix of nodes,nodes={draw, nodes={draw}},nodes in empty cells]{
         \\ \\ \\ \\
     };

    \draw[-latex] (0.25,-1) .. controls (0.25,-1.25) and (1,-1.25) .. (out.north);
    \draw[-latex] (in.south) .. controls (-1,1.5) and (-0.25,1.5) .. (-0.25,1);
\end{tikzpicture}
\end{center}

\begin{flushleft}
    Im Abitur wird eine Queue als Java-Klasse vorgegeben. \\
    \textattachfile[color=0 0 0]{code/Queue.java}{Abiturvorgabe: Queue}
\end{flushleft}

\subsection{Stack}
